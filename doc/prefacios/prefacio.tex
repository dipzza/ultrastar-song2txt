\thispagestyle{empty}

\begin{center}
\large\bfseries
UltraStar: \texttt{\normalfont{\large\bfseries Song2txt}}

Adaptación automática de canciones para juegos de karaoke
\end{center}

\begin{center}
Francisco Javier Bolívar Expósito
\end{center}


\vspace{0.5cm}
\noindent\textbf{Palabras clave}: \textit{Software libre, UltraStar, Python, Desarrollo ágil}
\vspace{0.7cm}

\noindent\textbf{Resumen}

La adaptación de canciones para videojuegos de karaoke es un proceso complejo y manual que requiere conocimientos musicales, lo que provoca que el número de canciones disponibles sea reducido. Aprovechando los avances en inteligencia artificial, gracias a los cuales es posible resolver problemas con sistemas informáticos que anteriormente requerían la intervención de un humano; este trabajo tiene el propósito de facilitar la adaptación de canciones avanzando en su automatización.

Para ello, utilizando un desarrollo ágil, se ha implementado una aplicación en Python que a partir del audio de una canción estima los tonos de cada nota interpretada por el cantante y los guarda en un archivo con el formato estándar UltraStar para videojuegos de karaoke, con una precisión superior a soluciones existentes. Adicionalmente, la librería desarrollada para trabajar con el formato UltraStar puede ser usada independientemente por otros proyectos.


\cleardoublepage

\begin{otherlanguage}{english}


\begin{center}
\large\bfseries 
\texttt{\normalfont\large\bfseries UltraStar: Song2txt}
	
Automatic adaptation of songs for karaoke games
\end{center}

\begin{center}
	\texttt{\normalfont Francisco Javier Bolívar Expósito}
\end{center}


\vspace{0.5cm}
\noindent\textbf{Keywords}: \textit{FLOSS, UltraStar, Python, Agile development}
\vspace{0.7cm}

\noindent\textbf{Abstract}

Adapting songs for karaoke video games is a complex and manual process that requires musical knowledge, which results in a limited number of songs available. Taking advantage of advances in artificial intelligence, thanks to which it is possible to solve problems with computer systems that previously required the intervention of a human; this work aims to facilitate the adaptation of songs by advancing in its automation.

To this end, using agile development, a Python application has been implemented that, from the audio of a song, estimates the tones of each note performed by the singer and saves them in a file with the standard \texttt{UltraStar} format for karaoke video games, with greater precision than existing solutions. Additionally, the library developed to work with the \texttt{UltraStar} format can be used independently by other projects.

\end{otherlanguage}

\cleardoublepage

\thispagestyle{empty}

\noindent\rule[-1ex]{\textwidth}{2pt}\\[4.5ex]

D. \textbf{Juan Julián Merelo Guervós}, Profesor del departamento de Arquitectura y Tecnología de Computadores.

\vspace{0.5cm}

\textbf{Informo:}

\vspace{0.5cm}

Que el presente trabajo, titulado \textit{\textbf{UltraStar: Song to txt}},
ha sido realizado bajo mi supervisión por \textbf{Francisco Javier Bolívar Expósito}, y autorizo la defensa de dicho trabajo ante el tribunal
que corresponda.

\vspace{0.5cm}

Y para que conste, expiden y firman el presente informe en Granada a junio de 2022.

\vspace{1cm}

\textbf{El/la director(a)/es: }

\vspace{5cm}

\noindent \textbf{Juan Julián Merelo Guervós}

\chapter*{Agradecimientos}

Aprovechando el fin de esta etapa académica, me gustaría agradecer a las personas que me han apoyado y gracias a las cuales soy quien soy.

En primer lugar, quiero agradecer el cariño de mi familia, especialmente a mis padres por su apoyo incondicional y a mi hermana por entenderme como nadie y por las pequeñas llamadas de 5 minutos que acaban durando horas.

También quiero dar las gracias a tantos profesores y compañeros de la universidad que no han dudado en prestar su ayuda, enseñándome tanto y compartiendo su pasión por este campo del conocimiento. A su vez, a la comunidad de software libre y a todas las personas que aportan desinteresadamente su granito de arena, confiando en llegar más lejos cooperando que compitiendo. 

Por último, a los \texttt{\normalfont\textit{Mákinas}} y los\texttt{\normalfont\textit{ Checolegas}}, la gente con la que he pasado algunos de los mejores momentos de esta etapa, con un gran corazón y con la que me siento en mi sitio.

Es imposible expresar todo lo que me habéis dado estos años y lo feliz que soy de teneros en mi vida. Un abrazo muy grande para todos.

\vspace*{1cm}

