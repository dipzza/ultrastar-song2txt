\chapter{Introducción}

Este proyecto es software libre, y está liberado con la licencia AGPLv3\cite{agpl}.

\section{Motivación}

Existen varios vídeo juegos de Karaoke de software libre con cierta popularidad (\href{https://github.com/UltraStar-Deluxe/USDX}{usdx}, \href{https://github.com/Vocaluxe/Vocaluxe}{vocaluxe}, \href{https://github.com/performous/performous}{performous}, ...) donde los jugadores consiguen puntos haciendo coincidir al cantar su tono con el de la canción original.

Para añadir una canción es necesario crear un archivo de texto con un formato estándar (el de UltraStar) que contiene metadatos de la canción, la letra separada en fonemas y el segundo y tono en el que se canta cada fonema.

Este proceso de añadir canciones es complejo y largo, especialmente si se carece de la experiencia musical para averiguar el tono de la voz.

Por este motivo, este trabajo pretende la creación de herramientas que automaticen parte de la creación de los archivos de texto.

Estas herramientas serían de gran ayuda para agilizar el trabajo de los usuarios que ya crean canciones de forma regular y también para reducir la barrera de entrada y permitir a usuarios sin conocimiento musical añadir canciones.

\section{Estado del Arte}

Aunque el archivo de texto podría crearse a mano normalmente se hace con ayuda de programas dedicados. El proceso más cómodo es usando dos programas como se describe a continuación.

\href{https://github.com/UltraStar-Deluxe/UltraStar-Creator}{UltraStar Creator} hace un proceso guiado con las siguientes funciones:

\begin{itemize}
	\item Extrae metadatos si seleccionas un mp3, te pide rellenar los que falten
	\item Detecta automáticamente el BPM de la canción
	\item Pide la letra de la canción y ofrece separar los fonemas según el idioma automáticamente
	\item Permite añadir el segundo de cada fonema escuchando la canción a distintas velocidades y pulsando la barra espaciadora cada vez que escuches un nuevo fonema
\end{itemize}

Tras este proceso se obtiene un archivo de texto base para trabajar sobre el, después se puede usar \href{https://github.com/sarutasan72/Yass}{Yass} que posee una mejor interfaz para afinar la sincronización de los fonemas en el tiempo y asignar el tono de cada fonema, la parte que consume más tiempo.


\section{Objetivos}

\subsection{Principal}

\textbf{Automatizar la detección de las notas de cada fonema}. Es la parte más compleja y que más tiempo requiere, por lo que sería la más productiva de automatizar.

\subsection{Opcionales}

Automatizar los siguientes procesos

\begin{itemize}
	\item Completar metadatos
	\item Obtener letra
	\item Separar fonemas según idioma
	\item Estimar BPM
\end{itemize}

Se dejan como secundarios ya que o son fáciles de realizar manualmente, o ya se encuentran automatizados en UltraStar Creator.

No se fija como objetivo la sincronización automática de los fonemas en el tiempo, aunque sería interesante para un trabajo futuro.