\chapter{Introducción}

Este proyecto es software libre, liberado con la licencia AGPLv3 \cite{agpl} y publicado en un repositorio público\footnote{\url{https://github.com/dipzza/ultrastar-song2txt}}.

\section{Motivación}
\label{sec:motivation}

El karaoke es un tipo de entretenimiento consistente en cantar junto a música grabada, normalmente con la ayuda de una pantalla en la que se muestra la letra de la canción e indicaciones de cuando hay que cantar cada parte.

Mucha gente disfruta del karaoke como forma de ocio o diversión, cumpliendo el sueño de ser cantante por unos momentos. Por esta razón, existen multitud de programas dirigidos a los entusiastas del karaoke.

Un tipo de estos programas son los videojuegos de karaoke, que ofrecen todavía más ayuda al mostrar el tono en el que se está cantando y como de lejos se encuentra este del correcto. Además, al medir la precisión con la que se ha cantado la canción permiten competir con otras personas o disfrutar buscando la mejora personal.

Sin embargo, los aficionados de los videojuegos de karaoke tienen que lidiar frecuentemente con un problema cuando quieren cantar su canción favorita: que no esté disponible.

Aunque el desarrollo de programas de software libre y gratuitos tanto de videojuegos de karaoke (\href{https://github.com/UltraStar-Deluxe/USDX}{\textit{UltraStar Deluxe}}, \href{https://github.com/Vocaluxe/Vocaluxe}{\textit{Vocaluxe}}, \href{https://github.com/performous/performous}{\textit{Performous}}...) como de herramientas para la creación de canciones (\href{https://github.com/sarutasan72/Yass}{\textit{Yass}}, \href{https://github.com/UltraStar-Deluxe/UltraStar-Creator}{\textit{UltraStar Creator}}...) ha ayudado al desarrollo de comunidades donde se crean y comparten canciones para videojuegos de karaoke, la mayoría de canciones siguen sin estar disponibles, especialmente aquellas más recientes o menos famosas.

Una causa de que la selección de canciones disponibles sea tan limitada es que el proceso de creación de canciones para videojuegos de karaoke es complejo, largo y requiere conocimientos musicales.


\section{Objetivo}

Este trabajo persigue ayudar en la creación de canciones para videojuegos de karaoke, reduciendo la barrera de entrada para personas sin conocimientos musicales y el tiempo que necesitan invertir aquellas con conocimientos musicales.


Para ello se pretende \textbf{mejorar o integrar los procesos de automatización existentes para la creación de canciones para videojuegos de karaoke.}

\section{Terminología}

\begin{itemize}
	\item{Nota. Representación musical de un sonido, con un tono constante y una duración determinada.}
	\item{Tono. Cualidad de los sonidos que permite su ordenación en una escala relacionada con la frecuencia.}
	\item{Frecuencia fundamental. Aquella que representa el tono simple más bajo presente en una nota.}
	\item{MIDI. Estándar técnico en el que se definen 128 notas por su número, con equivalencias a la nomenclatura estándar inglesa.}
	\item{Pulsación. Unidad básica en música para medir el tiempo. Una sucesión constante divide el tiempo en partes iguales.}
	\item{Pulsaciones por minuto. Unidad empleada para medir el ritmo en música. Equivale al número de pulsaciones que caben en un minuto.}
	\item{Fonema. Unidad sonora mínima que puede distinguir una palabra de otra en un lenguaje dado.}
	\item{Software libre. Aquel cuyo código fuente puede ser estudiado, modificado, utilizado libremente para cualquier propósito y redistribuido con cambios o mejoras.}
	\item{Software privativo. Aquel del cual no existe una forma libre de acceso a su código fuente, en el que uno o varios de los derechos de usar, modificar, compartir modificaciones o el software están restringidos.}
\end{itemize}