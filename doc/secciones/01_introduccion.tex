\chapter{Introducción}

Este proyecto es software libre, y está liberado con la licencia AGPLv3 \cite{agpl}.

\section{Motivación}

El karaoke es un tipo de entretenimiento consistente en cantar junto a música grabada, normalmente con la ayuda de una pantalla en la que se muestra la letra de la canción e indicaciones de cuando hay que cantar cada parte.

Mucha gente disfruta del karaoke como forma de ocio o diversión, cumpliendo el sueño de ser cantante por unos momentos. Por esta razón, existen multitud de programas dirigidos a los entusiastas del karaoke.

Un tipo de estos programas son los videojuegos de karaoke, que ofrecen todavía más ayuda al mostrar el tono en el que se está cantando y como de lejos se encuentra este del correcto. Además, al medir la precisión con la que se ha cantado la canción permiten competir con otras personas o disfrutar buscando la mejora personal.

Sin embargo, los aficionados de los videojuegos de karaoke tienen que lidiar frecuentemente con un problema cuando quieren cantar su canción favorita: que no esté disponible.

Aunque el desarrollo de programas de software libre y gratuitos tanto de videojuegos de karaoke (\href{https://github.com/UltraStar-Deluxe/USDX}{\textit{UltraStar Deluxe}}, \href{https://github.com/Vocaluxe/Vocaluxe}{\textit{Vocaluxe}}, \href{https://github.com/performous/performous}{\textit{Performous}}...) como de herramientas para la creación de canciones (\href{https://github.com/sarutasan72/Yass}{\textit{Yass}}, \href{https://github.com/UltraStar-Deluxe/UltraStar-Creator}{\textit{UltraStar Creator}}...) ha ayudado al desarrollo de comunidades donde se crean y comparten canciones para videojuegos de karaoke, la mayoría de canciones siguen sin estar disponibles, especialmente aquellas más recientes o menos famosas.

Una causa de que la selección de canciones disponibles sea tan limitada es que el proceso de creación de canciones para videojuegos de karaoke es complejo, largo y requiere conocimientos musicales.


\section{Objetivo}

Este trabajo persigue ayudar en la creación de canciones para videojuegos de karaoke, reduciendo la barrera de entrada para personas sin conocimientos musicales y el tiempo que necesitan invertir aquellas con conocimientos musicales.


Para ello se pretende \textbf{mejorar o integrar los procesos de automatización existentes para la creación de canciones para videojuegos de karaoke.}