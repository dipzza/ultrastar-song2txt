\chapter{Descripción del problema y estado del arte}

\section{Descripción del problema}

Con el fin de facilitar el proceso de añadir canciones a videojuegos de karaoke, se pretende avanzar en la automatización de las tareas relacionadas.

Independientemente del formato utilizado para guardar las canciones, las principales tareas consisten en:

\begin{itemize}
	\item{Encontrar e introducir metadatos, como el título, el artista, las pulsaciones por minuto, etc.}
	\item{Obtener la letra de la canción.}
	\item{Definir las notas interpretadas por el cantante original, con la siguiente información asociada a cada una:}
	\begin{itemize}
		\item{El tono del sonido.}
		\item{El intervalo de tiempo en el que sucede.}
		\item{Los fonemas de la letra correspondientes.}
	\end{itemize}
\end{itemize}

Algunas de las tareas son más complejas o tardan más en completarse. Definir la información de las notas en general es un proceso tedioso debido a la gran cantidad de notas que pueden ser cantadas en una canción (por ejemplo, la canción \texttt{'Rolling in the Deep' de Adele} requiere la definición de más de 500 notas individuales) y, en particular, la asignación de los tonos es especialmente compleja porque requiere de oído musical y un proceso de prueba y error.

¿Debe darse prioridad a la automatización de estas tareas, por ser las más complejas? Es un argumento a tener en cuenta, sin embargo, es importante valorar las soluciones ya existentes para decidir el enfoque del trabajo, analizadas en el capítulo \ref{ch:artstate} sobre el estado del arte.

